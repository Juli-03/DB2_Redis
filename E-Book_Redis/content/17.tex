%!TEX root = ../dokumentation.tex

\subsection{Vorteile}

\ac{Redis} ist gemacht für die Speicherung dynamischer und unstrukturierter Daten (vgl. \cite{4}, S.10). Da Nutzer die Auswahl aus zahlreichen Datentypen für die Modellierung ihrer Daten haben, ist \ac{Redis} auch für die Speicherung komplexer Datentypen geeignet (vgl. \cite{4}, S.11).

Wenn Daten sehr oft aktualisiert werden, sind relationale Datenbanken, wie \textit{MSSQL}, \textit{MySQL} oder \textit{Oracle} ungeeignet (vg. \cite{4}, S.10). Redis eignet sich für diesen Anwendungsfall, da es eine In-Memory-Datenbank ist. Dadurch entsteht eine geringe interne Komplexität, ebenso wie der größte Vorteil von \ac{Redis}: Die Performance, bzw. Geschwindigkeit.

Interessant ist auch, dass eine \ac{Redis}-Instanz nur 1MB an Arbeitsspeicher verbraucht, um zu laufen (vgl. \cite{4}, S.11). Der Overhead für Metadaten, wie z.B. Informationen über Datentypen ist sehr gering. Dadurch kann der Großteil des Arbeitsspeichers für die tatsächliche Speicherung der Daten verwendet werden.

Außerdem hat \ac{Redis} eine sehr ausführliche \href{https://redis.io/docs/}{Dokumentation} mit einem eigenen \href{https://www.youtube.com/@Redisinc}{YouTube-Kanal}. Dort werden Basis-Konzepte anschaulich erklärt. Dadurch wird der Einstieg in die Entwicklung mit \ac{Redis} vereinfacht und Entwickler erhalten kompetente Ünterstützung bei Problemen. Auch Wertkzeuge, wie \gls{ri} (siehe \autoref{subsec:tools}) machen die Entwicklung mit \ac{Redis} leichter.