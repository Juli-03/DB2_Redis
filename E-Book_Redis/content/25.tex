%!TEX root = ../dokumentation.tex

\section{Verwendete Technologien und Setup}
Im folgenden Abschnitt soll die Umsetzung der Chat-App genauer erläutert werden.
Die Umsetzung hängt stark von den verwendeten Technologien ab, welche im Nachfolgenden genau benannt werden.

Wie bereits in der Architektur der Anwendung genannt, besteht die Anwendung aus zwei Hauptkomponenten: dem Backend und dem Frontend (vgl. Abbildung \ref{fig:arch}).
Als Programmiersprache, um die Klassen des Backends zu implementieren, wurde die multiparadigma- Sprache \textit{Python} gewählt.
\textit{Python} unterstützt objektorientierte Programmierung und ermöglicht es, alle Klassen des Models zu modellieren. 
Zusätzlich bietet Python die Möglichkeit, verschiedenste Bibliotheken einzubinden.
Nennenswert sind die Bibliotheken \textit{redis} und \textit{flask}, welche die Grundfunktionalitäten der Anwendung bestimmen. 
Die \textit{redis} Bibliothek ermöglicht die Einbindung einer \acs{Redis}-Datenbank, welche Teil des Backends ist.

Das Frontend der Applikation wird durch \textit{flask} realisiert. Hierbei handelt es sich um ein Webframework.
\textit{Flask} ermöglicht es, in Verbindung mit \ac{HTML}, eine bidirektionale Verbindung zwischen dem Backend und dem Frontend zu realisierten.
Die \acs{HTML}-Templates erzeugen die \acs{GUI}, welche vom User bedient wird. Die Kommunikation in das Backend kann über \textit{POST} und \textit{GET}  Befehle ausgeführt werden, aber auch durch die Implementierung eines Websockets.

