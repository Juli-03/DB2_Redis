%!TEX root = ../dokumentation.tex

\section{Verwendete Technologien und Setup}
Im folgenden Abschnitt soll die Umsetztung der Chat-App genauer erläutert werden.
Die Umsetzung hängt stark von den verwendeten Technologien ab, welche im Nachfogenden genau benannt werden.

Wie bereits in der Architektur der Anwendung genannt, besteht die Anwendung aus zwei Hauptkomponenten: dem Backend und dem Frontend (vgl. Abbildung \ref{fig:arch}).
Als Programmiersprache um die Klassen des Backends zu implementieren, wurde die multiparadimatische Programmiersprache Python gewählt.
Python unterstützt objektorientierte Programmierung und ermöglicht es alle Klassen des Models zu modelieren. 
Zusätzlich bietet Python die Möglichkeit verschiedenste Biblioteken einzubinden.
Nennswert sind die Bibliotheken \textit{redis und flask}, welche die Grundfunktionalitäten der Anwendung bestimmen. 
Die \textit{redis} Bibliothek ermöglicht die Einbindung einer \acs{Redis}-Datenbank, welche Teil des Backends ist.

Das Frontend der Applikation wird durch \textit{flask} realisiert. Hierbei handelt es sich um ein Webframework.
Flask ermöglicht es, in Verbindung mit HTLM, eine bidirektionale Verbindung zwischen dem Backend und dem Frontend zu realisierten.
Die HTML-Templates erzeugen die Graphischebenutzeroberfläche, welche vom User bedient wird. Die Kommunikation ins Backend über kann über \textit{POST und GET} Befehle ausgeführt werden, aber auch durch die Implementierung eines Websockets.

