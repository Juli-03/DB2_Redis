%!TEX root = ../dokumentation.tex

\subsection{Skalierung}
\label{sec:Skalierung}
Wie bereits in \autoref{sec:Client-Server} angesprochen, bietet Redis die Möglichkeit mehrere Server zu verwenden.
Diese Server sind mit Hilfe einer \textit{Master-Slave}-Architektur miteinander verbunden.

Der Master-Server verwaltet weiterhin die gesammte Datenbank und empfängt Schreibanfragen der Clients.
Wenn Änderungen in der Datenbank erfolgen, werden diese Änderungen an die Slave-Server weitergegeben.
Das bietet den Vorteil, dass im Falle eines Ausfalls des Master-Knotens die Daten zum Einen nicht verloren gehen und zum Anderen schnell wieder zur Verfügung stehen.
Außerdem können die Slave-Knoten auch für Leseanfragen, wodurch die Last bei hohen Zugriffszahlen auf mehrere Server verteilt werden kann \cite[141]{3}.

Sollte einer der Slave-Knoten vorrübergehend ausfallen bzw. nicht erreichbar sein und deshalb keine Änderungen empfangen können, findet eine automatische Synchronisation statt, sobald der Slave-Knoten wieder erreichbar ist.
Der Server erkennt anhand der fehlenden Antwort des Slaves, dass dieser nicht erreichbar ist und speichert die Änderungen in einer Warteschlange, dem sogenannten \textit{Backlog-Buffer}. 
Dieser Backlog-Buffer hat eine feste Größe, die bei der Konfiguration des Servers festgelegt werden kann. 

Übersteigt die Größe der Änderungen während der Abwesenheit des Slaves die des Buffers, findet ein sogenannter \textit{Full-Resync} statt.
Das bedeutet, dass der Slave-Knoten mit einem aktuellen Snapshot der Daten aus dem Master neu initialisiert wird \cite{Redis-Docs-Replication}.
Andernfalls findet ein sogenannter \textit{Partial-Resync} statt. Dabei werden die Änderungen des Backlog-Buffers an den Slave-Knoten übertragen und dort ausgeführt \cite[146 - 153]{3}.

Die Größe des Backlog-Buffers sollte anhand des erwarteten Datenaufkommens festgelegt werden. 
Bei größeren Datenmengen sollte der Buffer größer gewählt werden, damit nicht schon bei kurzen Ausfallzeiten der Buffer überläuft und anschließend ein gesammter Snapshot übertragen werden muss.
Trotzdem sollte der Buffer nicht zu groß gewählt werden, da die Synchronisation mittels Full-Resync bei großen Datenmengen performanter sein kann, als die Übertragung der Änderungen mittels Partial-Resync.

Neben einer einfachen Kopie der Daten auf mehrere Server, bietet Redis auch die Möglichkeit die Datenbank auf mehrere Server zu verteilen.
Dieses Art der Erweiterung wird auch als horizontale Skalierung bezeichnet \cite{Fasel2016}.
Hierfür gibt es zwei verschiedene Ansätze die Daten auf mehrere Server aufzuteilen:
\begin{itemize}
    \item \textbf{Horizontale Partitionierung:} Unterschiedliche Keys werden auf unterschiedliche Server verteilt. Damit hat immer ein Server alle Daten eines bestimmten Keys \cite[148-149]{12}. 
    Beispielsweise sind alle Daten des Keys \textit{User} auf einem Server gespeichert und alle Daten des Keys \textit{Product} auf einem anderen Server. 
    \item \textbf{Vertikale Partitionierung:} Die Values eines Keys werden auf unterschiedliche Server verteilt \cite[148-149]{12}. 
    Ein Beispiel hierfür, wäre die Aufteilung des Keys \textit{User} auf mehrere Server. Server 1 speichert die ersten 10000 User, Server 2 die nächsten 10000 User und so weiter.
\end{itemize}



