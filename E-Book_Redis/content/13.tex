%!TEX root = ../dokumentation.tex

\subsection{Datentypen}
\label{subsec:datentypen}

Datentypen in \ac{Redis} sind die Strukturen, in denen Nutzer ihre Daten speichern können. Sie sind der Value, der einem Key zugeordnet wird. Jeder Datentyp hat emblematische Eigenschaften und Befehle, um mit den Daten, die im Datentyp gespeichert sind, umzugehen. In der \ac{Redis}-Dokumentation findet sich eine \href{https://redis.io/commands}{Liste aller Befehle} für alle Datentypen. Im Folgenden liegt der Fokus auf den Eigenschaften der verschiedenen Datentypen. 

Der einfachste Value, den man in \ac{Redis} einem Key zuordnen kann, ist der \textbf{String} (vgl. \cite{Redis-Docs-String}). Ein String-Wert kann bis zu 512 MB Daten beinhalten und wird unter anderem verwendet, um dynamische Strukturen, wie Listen und Sets zu realisieren (vgl. \cite{4}, S.36-39). Ein String kann z.B. ein Text oder eine Zahl sein. Auf Zahlen können dabei Integer-Operationen, wie das Inkrementieren angewendet werden (vgl. \cite{nosql}, S.86).

Eine Menge an Strings wird in \ac{Redis} als \textbf{Liste} bezeichnet. Diese wird intern als verkettete Liste umgesetzt, wobei die Reihenfolge der Strings duch die Einfüge-Reihenfolge bestimmt wird (vgl. \cite{nosql}, S.86). Der Vorteil ist, dass das Hinzufügen neuer Elemente immer gleich lang dauert (vgl. \cite{4}, S.36-39). Der Nachteil ist, dass ein Element im innern der Liste nur in $\mathcal{O}(N)$ Zeitkomplexität erreicht werden kann, wobei $N$ der Index des Elements in der Liste ist (vgl. \cite{4}, S.36-39). Listen werden in der Regel für Stacks oder Queues verwendet (vgl. \cite{Redis-Docs-List}). Die maximale Anzahl an Elementen in der Liste beträgt $2^{32} - 1$ (vgl. \cite{4}, S.36-39).

Ein \textbf{Set} ist eine ungeordnete Menge an Strings (vgl. \cite{4}, S.36-39). Im Set sind keine Duplikate erlaubt (vgl. \cite{nosql}, S.86). Die maximale Anzahl der Elemente im Set ist, wie bei Listen, auf $2^{32} - 1$ Elemente begrenzt (vgl. \cite{4}, S.36-39). Der Vorteil gegenüber Listen ist, dass der Anwender auf Elemente im Set lesend oder schreibend in $\mathcal{O}(1)$ zugreifen kann (vgl. \cite{4}, S.36-39). Sets werden häufig genutzt, um Indexe zu speichern (vgl. \cite{4}, S.36-39) oder eine Menge in der die Strings sowieso einzigartig sind (vgl. \cite{Redis-Docs-Set}).

