%!TEX root = ../dokumentation.tex

\subsection{Installation und Betriebssysteme}

\acs{Redis} ist in \textit{ANSI C} geschrieben und funktioniert somit auf den meisten \acs{POSIX} Systemen, wie z.B. \textit{Linux} oder \textit{macOS} (vgl. \cite{Redis-Docs-Intro}). \acs{POSIX} steht für \textit{Portable Operating System Interface} und umfasst eine Menge an Standards für \textit{UNIX}-Derivate (vgl. \cite{15}, S.8). In der offiziellen \ac{Redis}-Dokumentation heißt es: \glqq \textit{we recommend using Linux for deployment}\grqq (\cite{Redis-Docs-Intro}), die Entwickler empfehlen also \textit{Linux}, statt \textit{macOS}.

Es gibt keine offizielle Unterstützung für \textit{Windows} (vgl. \cite{Redis-Docs-Intro}). Allerdings ist es möglich, \ac{WSL2} auf \textit{Windows} zu installieren. Damit können \textit{Linux}-Binärdateien auf \textit{Windows} ausgeführt werden (vgl. \cite{Redis-Docs-Install-W}). Alternativ kann auch \textit{Ubuntu} auf \textit{Windows} virtualisiert werden. Es existieren unoffizielle \ac{Redis}-Versionen für \textit{Windows}, die allerdings nicht der Produktionsqualität entsprechen, von Versionen für \textit{Linux} oder \textit{macOS} (vgl. \cite{4}, S.16). 

Für die Installation gibt es zwei Möglichkeiten. Entweder man installiert nur \ac{Redis} oder den \ac{Redis}-Stack (vgl. \cite{Redis-Docs-Install-R-or-RS}). Der \ac{Redis}-Stack erweitert den standard \ac{Redis}-Download um folgende Features (vgl. \cite{Redis-Docs-Redis-Stack}):

\begin{itemize}
	\item \ac{JSON}
	\item \textit{Search and Query}
	\item \textit{Time Series}
	\item \textit{Bloom Filter}
\end{itemize}

\ac{JSON}, \textit{Time Series} und \textit{Bloom Filter} sind \ac{Redis}-spezifische Datentypen, die in Kapitel TODO erklärt werden. \textit{Search and Query} ist eine Reihe an Möglichkeiten, um einfache und effiziente Abfragen an den \ac{Redis}-Server zu stellen (vgl. \cite{Redis-Docs-Search-Query}). Beispielsweise ist mit \textit{Search and Query} eine simple Indexierung von \ac{JSON}-Objekten möglich (vgl. \cite{Redis-Docs-Search-Query}). Ebenfalls kann der Nutzer damit Aggregationen durchführen, wie z.B. das Gruppieren oder Sortieren von Abfrageergebnissen (vgl. \cite{Redis-Docs-Aggregation}).  

Unter \textit{Linux} lassen sich \ac{Redis} oder der \ac{Redis}-Stack über einen der folgenden Terminal- befehle installieren:

\texttt{sudo apt-get install redis}

\texttt{sudo apt-get install redis-stack-server}

\vspace{0.4cm}

Auf \textit{macOS} können \ac{Redis} und der \ac{Redis}-Server einfach per \textit{brew} installiert werden:

\texttt{brew install redis}

\texttt{brew install redis-stack}

\vspace{0.4cm}

Die Installation kann mit folgendem Befehl getestet werden:

\texttt{redis-server}

Wenn die Installation erfolgreich war, wird der \ac{Redis}-Server gestartet. Dieser kann mit der Tastenkombination \textit{Ctrl + C} gestoppt werden. Alternativ ist der Befehl 

\texttt{redis-cli shutdown} 

Dabei stoppt der \ac{Redis}-Server erst alle Clients, sorgt dann für Persistenz und beendet erst anschließend den Prozess (vgl. \cite{3}, S.13ff.). Beim klassischen \textit{kill}-Befehl können Daten verloren gehen, da vor dem \textit{Shutdown} keine Persistenz-Behandlung durchgeführt wurde (vgl. \cite{3}, S.13ff.).

\ac{Redis} kann in verschiedenen Frameworks und Programmiersprachen genutzt werden. Hierfür gibt es vier \ac{Redis}-\acs{OM}-libraries (vgl. \cite{Redis-Docs-Redis-Stack}): 

\begin{itemize}
	\item Redis OM .NET
	\item Redis OM Node
	\item Redis OM Python
	\item Redis OM Spring
\end{itemize}

\ac{OM} übernehmen in \ac{Redis} das \ac{ORM}. \ac{ORM} ist eine Menge an Techniken, um inkompatible Systeme eines Computers miteinander zu verbinden (vgl. \cite{16}, S.1).





