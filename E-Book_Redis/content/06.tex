%!TEX root = ../dokumentation.tex

\subsection{In Memory}
\label{subsec:inMemory}

\acs{Redis} zählt zu den sogenannten \ac{IMDB}. 
Das bedeutet, dass alle Daten zur Laufzeit im \ac{RAM} des Servers liegen.
Dieses Vorgehen verhilft \acs{Redis} zu seinem größten Vorteil: der Geschwindigkeit.

Denn im Vergleich zu anderen Datenbanksystemen, deren Datenspeicherung üblicherweise auf Festplatten stattfindet, ist das Lesen und Schreiben von Daten in den \acs{RAM} um ein vielfaches schneller (vgl. \cite{VL_Rechnerarchitektur}).
Dadurch ist \acs{Redis} in der Lage, eine sehr hohe Anzahl an Operationen pro Sekunde zu verarbeiten.
Genaueres über die Performance wird später in \autoref{sec:Performance} erläutert.

\newpage Der Vorteil der Geschwindigkeit, den man sich durch die Verwendung einer \acs{IMDB} verschafft, ist jedoch mit zwei großen Nachteilen gekoppelt.
Der erste Nachteil ist rein monitärer Natur. 
Denn \acs{RAM} ist im Vergleich zu Festplattenspeicher sehr teuer. Für das gleiche Geld bekommt man nur einen Bruchteil an Speicherplatz.
Außerdem ist die maximale Größe des Arbeitsspeichers in einem Server aufgrund von technischen Einschränkungen begrenzt.

Der zweite Nachteil bezieht sich auf die Speichereigenschaften des \acs{RAM}s.
Dieser ist im Gegensatz zu Festplattenspeichern nicht persistent.
Das bedeutet, dass die Daten im \acs{RAM} nur solange erhalten bleiben, wie dieser mit Strom versorgt wird.
Fällt die Stromversorgung aus, gehen alle Daten aus dem \acs{RAM} verloren. 