%!TEX root = ../dokumentation.tex

\subsection{Nachteile}

Ein Nachteil bei der Entwicklung mit \ac{Redis} ist, dass das Speichern von großen Datenmengen sehr teuer ist. Arbeitsspeicher ist teurer, als gewöhnlicher Festplattenspeicher und \ac{Redis} ist eine In-Memory-Datenbank (siehe \autoref{subsec:inMemory}). Wenn Daten aus dem Arbeitsspeicher in den Hauptspeicher ausgelagert werden würden, ginge die hohe Geschwindigkeit von \ac{Redis} verloren. 

Für \textit{Windows}-Nutzer ist ebenfalls von Nachteil, dass es keine offizielle \ac{Redis}-Version für \textit{Windows} gibt. Allerdings können \ac{WSL2} oder eine virtuelle Maschine verwendet werden, um die \textit{Linux}-Version zu nutzen (siehe \autoref{subsection:installation}).

Außerdem erfüllt \ac{Redis} nicht die \gls{acid}-Eigenschaften, weshalb relationale Daten nicht mit \ac{Redis} gespeichert werden sollten (vgl. \cite{nosql}, S.91).