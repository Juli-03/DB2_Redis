%!TEX root = ../dokumentation.tex

\subsection{Anwendungsfelder}
Die Anwendungsfelder für eine \acs{Redis}-Datenbank sind vielfältig. Bei Anwendungen, welche eine hohe Performance benötigen, eignet sich \acs{Redis}. Wie aus den vorangegangen Abschnitten deutlich wurde, ist eine große Stärke von \acs{Redis} die Geschwindigkeit, die das System mit sich bringt (vgl. Abschnitt \ref{subsec:inMemory}).

Die offizielle \acs{Redis}-Dokumentation listet einige Anwendungsfelder. Darunter Indexierung, oder Volltextsuche. In diesem Anwendungsfall ist es möglich dies entweder in einer weiteren \acs{Redis}-Instanz, oder einer anderen Datenbank zu bewerkstelligen (vgl. \cite{Redis-Docs-Anwen}).
\\Weitere Anwendungsfälle sind Analysen. Diese sind vielfältig, beispielsweise die Analyse von Netzwerkkommunikationsdaten. Dabei wird der Netzverkehr aufgezeichnet und kann anschließend analysiert werden. Auch die Analyse von Kundenverhalten kann mit \acs{Redis} realisiert werden (vgl. \cite{Redis-Docs-Anwen}).

Neben Indexierung und Analysen bietet sich \acs{Redis} als Realisierung eines Cache an. Dieser kann beispielsweise in Webanwednungen eingesetzt werden, um oft genutzte Daten zu cachen und so schnell verfügbar zu machen (vgl. \cite{Redis-Docs-cache}).
\\
Ein weiterer Anwendungsfall, ist die Realisierung eines Scoreboards, beispielsweise in Online-Spielen. Aufgrund der Eigenschaften der Datentypen in \acs{Redis} kann eine solche Rangliste effizient realisiert werden. Der Score der Sorted-Sets sortiert die Einträge \newline(vgl. \cite{Redis-Docs-leaderboard}).
\\
\acs{Redis} bietet ebenfalls die Möglichkeit neben anderen Systemen und Datenbanken zu koexistieren. Beispielsweise kann eine relationale Datenbank eingesetzt werden, um Daten zu speichern, während \acs{Redis} für das Caching oft genutzter Daten zuständig ist.
