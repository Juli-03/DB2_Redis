%!TEX root = ../dokumentation.tex

\subsection{Anwendungsfelder}
Die Anwendungsfelder für eine \acs{Redis}-Datenbank sind vielfältig. Bei Anwendungen, welche eine hohe Performance benötigen, bietet sich \acs{Redis} gut an. Wie aus den vorangegangen Abschnitten deutlich wurde, ist eine große Stärke von \acs{Redis} die Geschwindigkeit, die das System mit sich bringt (vgl. Abschnitt \ref{subsec:inMemory}).

Die offizielle \acs{Redis}-Dokumentation listet einige Anwendungsfelder. Darunter Indexierung, oder Volltextsuche. In diesem Anwendungsfall ist es möglich dies entweder in einer weiteren \acs{Redis} Instanz, oder einer anderen Datenbank zu bewerkstelligen (vgl. \cite{Redis-Docs-Anwen}).
\\Weitere Anwendungsfälle sind Analysen. Diese sind vielfältig, beispielsweise die Analyse von Netzwerkkommunikationsdaten. Dabei wird der Netzverkehr aufgezeichnet und kann anschließend analysiert werden. Auch die Analyse von Kundenverhalten kann mit \acs{Redis} realisiert werden (vgl. \cite{Redis-Docs-Anwen}).

Neben Indexierung und Analysen bietet sich \acs{Redis} als Realisierung eines Cache an. Dieser kann beispielsweise in Webanwednungen eingesetzt werden, um oft genutzte Daten zu cachen und so schnell verfügbar zu machen (vgl. \cite{Redis-Docs-cache}).

Ein weiterer nicht, offensichtlicher Anwendungsfall, ist die Realisierung eines Scoreboards, beispielsweise in Online-Spielen. Aufgrund der Eigenschaften der Datentypen in \acs{Redis} kann eine solche Rangliste effizient realisiert werden. Die \textit{Sorted-Sets} in \acs{Redis} haben eine Ordnung (vgl. Abschnitt \ref{subsec:datentypen}). Mit eben jener Eigenschaft kann eine Rangliste realisiert werden (vgl. \cite{Redis-Docs-leaderboard}).

Da \acs{Redis} einfach aufzusetzen ist, bedarf es keiner Entscheidung zwischen dem einen oder anderen System (vgl. Abschnitt \ref{subsection:installation}). \acs{Redis} kann zusätzlich genutzt werden. Beispielsweise kann für eine Webanwendung eine SQL-Datenbank eingesetzt werden. Zusätzlich, um das Caching zu realisieren wird \acs{Redis} eingesetzt. Aufgrund der einfachen Installation und Aufsetzung kann \acs{Redis} flexibel zusätzlich eine Applikation integriert werden.
