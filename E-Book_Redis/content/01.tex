%!TEX root = ../dokumentation.tex

\chapter{Einführung}

Neben den klassischen relationalen Datenbanken setzen Entwickler heutzutage immer öfter auf \ac{NoSQL}-Datenbanken, um vor allem unstrukturierte Daten zu speichern. Dabei gibt es verschiedene Systeme mit jeweils emblematischen Eigenschaften und Funktionen. 

Eine aufstrebende und unter Entwicklern beliebte \ac{NoSQL}-Datenbank ist der \ac{Redis}. Zu dessen Besonderheiten gehört, dass es sich um eine \gls{kvDB} handelt mit unterschiedlichen Datentypen und Konfigurationsmöglichkeiten für die Positionierung im \acs{CAP}-Theorem.

Diese Arbeit umfasst die Geschichte von \ac{Redis}, sowie dessen Architektur und welche Möglichkeiten Entwickler haben, um ihre Projekte mit \ac{Redis} zu realisieren. Anschließend wird der \acl{Redis} in das \acs{CAP}-Theorem eingeordnet und seine Vor- und Nachteile werden evaluiert. Außerdem werden die Idee und Umsetzung einer beispielhaften Implementierung eines \ac{Redis}-Systems demonstriert, einer Online-Chat-Applikation. Das Ende dieser Arbeit fasst die gewonnenen Erkenntnisse zusammen und präsentiert eine kritische Reflexion des gesamten Themenbereichs um \ac{Redis}.