%!TEX root = ../dokumentation.tex

\section{Geschichte}
\label{Geschichte}
Edgar F. Codd definierte relationale Datenbanken bereits um 1970 (vgl. \cite{codd}). Bis heute sind diese eine etablierte Herangehensweise, um Daten zu speichern. Im letzten Jahrzehnt ist die Menge an zu verarbeitenden Daten exponentiell gestiegen (vgl. \cite{nosql}, S.7). Immer mehr dieser Daten sind unstrukturiert. Das bedeutet, dass Datensätze keiner festen Struktur unterliegen. Da relationale Datenbanken stets ein vorgeschriebenes Schema nutzen, sind unstrukturierte Daten nicht effizient speicherbar in einer solchen Datenbank (vgl. \cite{2016sql}, S.11). 
\\Aus der Notwendigkeit unstrukturierte Daten effizient zu speichern, entstanden \acp{NoSQL-Datenbank}. Anders als relationale Datenbanken besitzen diese kein festes Schema und sind nicht relational (vgl. \cite{2016sql}, S.222).

Der Informatiker \textit{Salvatore Sanfilippo} erkannte 2006 das gleiche Problem. In seinem Start-Up nutzte er zur Verarbeitung und Analyse von Webdateien ein\textit{ MySQL-System}. Da bei diesem Prozess größtenteils unstrukturierte Dateien verarbeitet werden mussten, suchte \textit{Sanfilippo} nach einer effizienteren Herangehensweise (vgl. \cite{nosql}, S.82).
\\Aus diesem Grund entwickelte \textit{Sanfilippo} eine eigene \acs{NoSQL-Datenbank}, die seinen Anforderungen entsprach. Ursprünglich hat \textit{Sanfilippo} die Datenbank in der Programmiersprache \textit{C} geschrieben.
\\2009 veröffentlichte \textit{Sanfilippo} seine Lösung erstmals. Seither ist das System Open-Source und als \acs{Redis} bekannt (vgl. \cite{learningRedis}, S.1). Heutzutage ist \acs{Redis} eine etablierte \acs{NoSQL-Datenbank}, die besonders für ihre Geschwindigkeit bekannt ist. Die Datenbank genießt hohes Ansehen unter Entwicklern. So befand sich \acs{Redis} 2023 in der \textit{Stack Overflow Developer Survey} auf dem sechsten Platz der beliebtesten Datenbank-Technologien (vgl. \cite{stackOver}). Bei dieser Umfrage handelt es sich um eine etablierte jährliche Auswertung der Meinungen von über 80.000 Entwicklern. Des Weiteren ergab diese Umfrage, dass ca. 11.000 Entwickler, die \textit{PostgreSQL} nutzen auf \acs{Redis} umsteigen möchten (vgl. \cite{stackOver}). Dies zeigt deutlich, dass \acs{Redis} unter Entwicklern beliebt und zeitgemäß ist. 