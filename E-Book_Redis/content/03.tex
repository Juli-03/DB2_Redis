%!TEX root = ../dokumentation.tex

\section{Geschichte}
Im folgenden Abschnitt soll der geschichtliche Werdegang des \ac{Redis} beleuchtet werden.
\\Edgar F. Codd definierte relationale Datenbanken bereits um 1970 (vgl. \cite{codd}). Bis heute sind diese eine etablierte Herangehensweise Daten zu speichern. Im letzten Jahrzehnt ist jedoch die Menge an zu verarbeitenden Daten exponentiell gestiegen (vgl. \cite{nosql}, S.7). Immer mehr dieser Daten sind unstrukturiert. Das bedeutet, dass Datensätze keiner festen Struktur unterliegen. Da relationale Datenbanken stets ein vorgeschriebenes Schema nutzen, sind unstrukturierte Daten nicht effizient speicherbar in einer solchen Datenbank (vgl. \cite{2016sql}, S. 11). 
\\Aus der Notwendigkeit unstrukturierte Daten effizient zu speichern, entstanden \ac{NoSQL-Datenbanken}. Anders als relationale Datenbanken besitzen diese kein festes Schema und sind nicht relational (vgl. \cite{2016sql}, S.222).

Der Informatiker \textit{Salvatore Sanfilippo} erkannte 2006 das gleiche Problem. In seinem StartUp nutzte er zu Verarbeitung und Analyse von Webdateien ein\textit{ MySQL-System}. Da bei diesem Prozess größtenteils unstrukturierte Dateien verarbeitet werden mussten, suchte \textit{Sanfilippo} nach einer effizienteren Herangehensweise (vgl. \cite{nosql}, S.82).
     