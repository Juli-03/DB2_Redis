%!TEX root = ../dokumentation.tex

\chapter{Kritische Reflexion und Ausblick}

Diese Arbeit zeigt, dass die \ac{NoSQL}-Datenbank \ac{Redis} Nutzern eine Menge an Möglichkeiten zur Verfügung stellt, um ihre Datenspeicherung zu modellieren. So bietet \ac{Redis} verschiedene Datentypen, Entwickler-Tools und Funktionen, wie z.B. Publisher und Subscriber. 

Außerdem profitieren \ac{Redis}-Entwickler von breiten Konfigurationsmöglichkeiten. So können Sicherheit, Persistenz und die Positionierung im \ac{CAP}-Theorem auf die Bedürfnisse des Anwenders angepasst werden.

Die Implementierung der Chat-Applikation demonstriert die Vielfältigkeit und Performance von \ac{Redis}. Die Anwendung zeigt zudem, wie einfach \ac{Redis} als In-Memory-Datenbank in Projekte, z.B. in \textit{Python}, integriert werden kann.

Allerdings gilt es auch festzuhalten, dass \ac{Redis} keine Allgemeinlösung für alle Projekte ist. Wenn der Fokus in einem Projekt auf der Performance liegt, ist \ac{Redis} eine gute Wahl. Sollte stattdessen eher vollständige Persistenz, oder die Einhaltung der \gls{acid}-Eigenschaften gefordert sein, kommt \ac{Redis} an seine Grenzen. 

Interessant ist auch der Ansatz, \ac{Redis} mit einem anderen Datenbanksystem zu verknüpfen, z.B. einer relationalen Datenbank. Dann können die Vorteile beider Systeme genutzt werden, um den optimalen Mehrwert für eine entsprechende Applikation zu bieten. 

Es lässt sich also festhalten, dass \ac{Redis} einer der Top-Kandidaten eines jeden Entwicklers sein sollte, wenn es um Schnelligkeit und Performance in Anwendungen geht. Genau dann erleben Entwickler, wie \ac{Redis} seiner Mission schon heute gerecht wird. Ganz nach dem eigenen Slogan \glqq \textit{Make apps faster}\grqq .