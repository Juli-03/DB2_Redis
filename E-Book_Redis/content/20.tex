%!TEX root = ../dokumentation.tex

\subsection{Antipattern}
Technologien haben etablierte Herangehensweisen, um Probleme zu lösen. Ebenso gibt es Herangehensweisen, die mehr Probleme erzeugen als lösen. Diese werden als \textit{Antipatterns} bezeichnet (vgl. \cite{antipattern}, S.1). Es wird davon abgeraten, diese zu nutzen. \acs{Redis} hat verschiedene nennenswerte \textit{Antipattern}.

Da es sich bei \acs{Redis} um eine In-Memory-Datenbank handelt, ist es essentiell, dass die Daten komplett in den Arbeitsspeicher geladen werden können (vgl. Abschnitt \ref{sec:inMemory}). Ist dies nicht gegeben, kann es zu massiven Performanceeinbrüchen kommen (vgl. \cite{Redis-Docs-antipattern}). Kommt ein Entwickler zu dem Punkt, dass der Arbeitsspeicher nicht ausreichend ist, kann die \acs{Redis}-Instanz skaliert werden. Hierbei gibt es verschiedene Skalierungsansätze, welche bereits in Abschnitt \ref{sec:Skalierung} erläutert wurden.

Ein weiteres \textit{Antipattern} ist die Ausführung von einzelnen Operationen. Die serielle Ausführung mehrerer Operationen erhöht den Verbindungs-Overhead. Aus diesem Grund soll eine Pipeline verwendet werden. Dabei werden mehrere Operationen ausgeführt, ohne auf die Antwort der einzelnen Operationen zu warten und die Antworten später zu verarbeiten, wenn sie eintreffen (vgl. \cite{Redis-Docs-antipattern}).

\acs{Redis} besitzt viele Einstellungsmöglichkeiten. Darunter sind Sicherheitseinstellungen. Standardmäßig sind diese nicht für ein Produktionssystem geeignet. Aus diesem Grund ist es wichtig die Sicherheitseinstellung anzupassen. Es ist ein \textit{Antipattern}, diese Einstellungen nicht anzupassen. Dazu gehören unter anderem die Vergabe eines Passwortes für die Datenbank, oder einer \gls{whitelist} (vgl. \cite{Redis-Docs-antipattern}).

Ein weiteres \textit{Antipattern} ist die Nutzung des \textbf{KEYS}-Befehls. Dieser vergleicht alle gespeicherten Schlüssel einer \acs{Redis}-Instanz. Die Ausführung dieses Befehls auf einer großen Instanz kann zu hohen Performanceeinbrüchen führen. Der Befehl ist vergleichbar mit dem Ausführen eines SQL-Befehls, ohne Bedingung (SELECT * FROM DB) (vgl. \cite{Redis-Docs-antipattern}).