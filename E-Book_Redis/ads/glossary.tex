%!TEX root = ../dokumentation.tex

%
% vorher in Konsole folgendes aufrufen:
%	makeglossaries makeglossaries dokumentation.acn && makeglossaries dokumentation.glo
%

%
% Glossareintraege --> referenz, name, beschreibung
% Aufruf mit \gls{...}
%

\newglossaryentry{acid}{name={ACID},description={bezeichnet die vier Schlüsseleigenschaften einer Transaktion: Atomicity (Atomizität), Consistency (Konsistenz), Isolation (Isolation) und Durability (Dauerhaftigkeit).}}

\newglossaryentry{kvDB}{name={Key-Value-Datenbank},plural={Key-Value-Datenbanken},description={Ausprägung einer NoSQL-Datenbank. Werte (Value) in Key-Value-Datenbanken werden über einen Schlüssel (Key) eindeutig identifiziert.}}

\newglossaryentry{hash}{name={Hashwert},description={bezeichnet die Ausgabe einer Hashfunktion, welche eine Abbildung von einer großen Definitionsmenge auf eine kleinere Zielmenge realisiert.}}

\newglossaryentry{ri}{name={RedisInsight},description={\acs{GUI} für die Visualisierung, Analyse und Manipulation von \ac{Redis}-Instanzen.}}