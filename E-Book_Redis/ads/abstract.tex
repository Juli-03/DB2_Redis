%!TEX root = ../dokumentation.tex

\pagestyle{empty}

\iflang{de}{%
% Dieser deutsche Teil wird nur angezeigt, wenn die Sprache auf Deutsch eingestellt ist.
\renewcommand{\abstractname}{\langabstract} % Text für Überschrift

% \begin{otherlanguage}{english} % auskommentieren, wenn Abstract auf Deutsch sein soll
\begin{abstract}
\textbf{Deutsch:}

Diese Dokumentation bietet eine umfassende Analyse des Remote Dictionary Servers (Redis), einer beliebten Key-Value NoSQL-Datenbank. Sie untersucht die Geschichte, Architektur und Funktionsweise von Redis, einschließlich seiner In-Memory-Fähigkeiten und Einordnung im CAP-Theorem. Die Vor- und Nachteile von Redis werden bewertet, und es wird eine beispielhafte Implementierung in Form einer Online-Chat-Applikation dargestellt. Abschließend erfolgt eine kritische Reflexion des Einsatzes von Redis.

\textbf{Englisch:}

This documentation presents a thorough analysis of the Remote Dictionary Server (Redis), a popular key-value NoSQL database. It explores the history, architecture, and functionality of Redis, including its in-memory capabilities and positioning within the CAP Theorem. The advantages and challenges of Redis are evaluated, and an exemplary implementation in the form of an online chat application is presented. Finally, a critical reflection on the use of Redis is offered.
\end{abstract}
% \end{otherlanguage} % auskommentieren, wenn Abstract auf Deutsch sein soll
}



\iflang{en}{%
% Dieser englische Teil wird nur angezeigt, wenn die Sprache auf Englisch eingestellt ist.
\renewcommand{\abstractname}{\langabstract} % Text für Überschrift

\begin{abstract}
Diese Dokumentation bietet eine umfassende Analyse des Remote Dictionary Servers (Redis), einer beliebten Key-Value NoSQL-Datenbank. Sie untersucht die Geschichte, Architektur und Funktionsweise von Redis, einschließlich seiner In-Memory-Fähigkeiten und Einordnung im CAP-Theorem. Die Vor- und Nachteile von Redis werden bewertet, und es wird eine beispielhafte Implementierung in Form einer Online-Chat-Applikation dargestellt. Abschließend erfolgt eine kritische Reflexion des Einsatzes von Redis.

\end{abstract}
}