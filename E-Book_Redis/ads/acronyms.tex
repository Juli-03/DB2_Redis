%!TEX root = ../dokumentation.tex

\addchap{\langabkverz}
%nur verwendete Akronyme werden letztlich im Abkürzungsverzeichnis des Dokuments angezeigt
%Verwendung: 
%		\ac{Abk.}   --> fügt die Abkürzung ein, beim ersten Aufruf wird zusätzlich automatisch die ausgeschriebene Version davor eingefügt bzw. in einer Fußnote (hierfür muss in header.tex \usepackage[printonlyused,footnote]{acronym} stehen) dargestellt
%		\acs{Abk.}   -->  fügt die Abkürzung ein
%		\acf{Abk.}   --> fügt die Abkürzung UND die Erklärung ein
%		\acl{Abk.}   --> fügt nur die Erklärung ein
%		\acp{Abk.}  --> gibt Plural aus (angefügtes 's'); das zusätzliche 'p' funktioniert auch bei obigen Befehlen
%	siehe auch: http://golatex.de/wiki/%5Cacronym
%	
\begin{acronym}[YTMMM]
\setlength{\itemsep}{1.5em} % Vertikaler Abstand zwischen den Acronyms
\setlength{\parskip}{1em}   % Vertikaler Abstand zwischen Absätzen

\acro{AGPL}{Affero GNU General Public License}
\acro{WSN}{Wireless Sensor Network}
\acro{MANET}{Mobile wireless Ad-hoc NETwork}
\acro{MAC}{Multiple Access Control}
\acro{QoS}{Quality of Service}
\acro{DSR}{Dynamic Source Routing}
\acro{API}{Application Programming Interface}
\acro{WYSIWYG}{What You See Is What You Get}
\acro{HTML}{HyperText Markup Language}

\acro{Redis}{Remote Dictionary Server}
\acro{NoSQL-Datenbank}{Not-only-SQL-Datenbank}
\acrodefplural{NoSQL-Datenbanken}{Not-only-SQL-Datenbanken}

\acro{CAP}{Consistency, Availability, Partition Tolerance}


\acro{POSIX}{Portable Operating System Interface}
\acro{WSL2}{Windows Subsystem for Linux}
\acro{JSON}{JavaScript Object Notation}
\acro{OM}{Object-Mapper}
\acro{ORM}{Object-Relational Mapping}
\acro{Redis-cli}{Redis command line interface}
\acro{GUI}{Graphical User Interface}
\acro{CRUD}{create, read, update, delete}
\acro{IMDB}{In-Memory-Datenbank}

%\DeclareAcronym{OM}{
%	short = OM,
%	long = Object-Mapper,
%	long-plural-form = Object-...
%}

\end{acronym}
